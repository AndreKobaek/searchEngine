\section{Inverted Indices}

\subsection{Implementation}\label{sec:invertedindicies-imp}
Based on the prototype for the project we implemented two inverted indices. The \code{InvertedIndexHashMap} and \code{InvertedIndexTreeMap} are two subclasses of the superclass \code{InvertedIndex} which in turn implement the Index interface. The inverted index is a clever implementation of the \code{SimpleIndex} class trying to overcome the problems given by this naive implementation. The \code{SimpleIndex} looks through every website in the database for any query given. The purpose of an inverted index is to preemptively create an index of websites to a word when the program is build, rather than performing the task for every query. In this scenario when a user makes a query the search will be performed on a \code{Map} object which maps words contained in the websites to the website objects. In this way we increase the time needed for building the application, but we decrease the time needed for every single query. But since the building only takes place ones, supposedly at a convenient time for the user, this is an optimal trade off.

The two subclasses have only one difference, they instantiate two different dynamic types of the map object. A \code{Map} is an interface which once implemented by a class will map a key to a value. The key needs to be unique but the value can be repeated.
The \code{InvertedIndexHashMap} inherits all the methods and variables from the abstract superclass \code{InvertedIndex} and instantiates a \code{HashMap}. The \code{HashMap} is an implementation of the Map interface, it maps a data value (in this case a \code{Collection<Website>}) to a specific key, in this case a string which represents one word presented in a specific \code{Website}. The \code{HashMap} does not follow any index and the order of the objects contained can change over time.
On the other hand we have the \code{InvertedIndexTreeMap}, which instantiates a \code{TreeMap} dynamic type for the map variable and inherits all the methods and variables from the abstract superclass \code{InvertedIndex}. The \code{TreeMap} follows the same behaviour as the \code{HashMap} with only one important difference, the \code{TreeMap} has an ordered index of its mappings. The \code{TreeMap} is sorted following the natural order of the keys or according to a provided \code{Comparator}. In our case this means that the \code{String} key will be sorted alphabetically.

As previously mentioned, the two classes \code{InvertedIndexTreeMap} and \code{InvertedIndexHashMap} extend the superclass \code{InvertedIndex} which is an abstract class, this means that is not possible to instantiate an \code{InvertedIndex} object directly. This structure allows us to write the code in a more structured way, avoiding duplicate code and leading to easy extensibility of the application. Hence, we implement all the common behaviour of an index into the \code{InvertedIndex} superclass and then we implement the details into the specific subclasses.

\subsection{Benchmarking}
The different index implementations of the were to be benchmarked and compared. 
The benchmarking process would look up a list of 20 different queries, on a given database file with a given index. The expected findings in terms of performance were that the \code{TreeMap} implementation would faster than the \code{HashMap} implementation on larger data file sizes. This expectation was due to \code{TreeMap} having a logarithmic growth in time cost of the \code{.get} method proportionate to map size, compared to the constant growth in time cost for the \code{HashMap}. \\
The efficiency of the two indices was then expected to intersect at some map size, with \code{HashMap} being the most efficient up until the intersection size and then the \code{TreeMap} would be the most efficient of sizes bigger than that. \\
The benchmark was applied on all the handed out file sizes and for all the available index implementations. The results of the benchmarking can be found in Table \ref{tab:benchmark:indices}.

\paragraph{Analysis and discussion:}
The results of the benchmarking did not align completely with our expectations. As can be seen from the results, the \code{InvertedIndexTreeMap} did not manage to outperform \code{InvertedIndexHashMap}. This was probably due to the data files not having a sufficient size for \code{TreeMap} to overshadow the \code{HashMap}. The performance for the two inverted indices were very similar, with the relative difference deteriorating as the data file sizes increase. Judging from the development in the differences it would take a significantly bigger data size for the \code{TreeMap} to be the better option. Since we do expect to end up using a much bigger data file, and the negligible difference between the two in general we decided to keep the \code{TreeMap} implementation in our code for now.

\begin{table}[t]
\centering
\begin{tabular}{llrr} \toprule
	Index					& File size & Avg LUT 	& $\pm$ Error \\ \midrule
	SimpleIndex				& tiny		& 22.92  	&	 0.76 \\ 
						 	& small		& 10832.42 	&	 2088.43 \\
							& medium 	& 283513.44 & 	 7717.45 \\ \hline
	InvertedIndexTreeMap	& tiny 		& 5.44 		& 	 0.29 \\
							& small 	& 9.28 		& 	 0.20  \\
							& medium 	& 135.26 	& 	 2.57 \\ \hline
	InvertedIndexHashMap	& tiny		& 4.68 		& 	 0.15 \\
							& small		& 8.16 		& 	 0.30 \\
							& medium 	& 120.40 	& 	 4.36 \\ \bottomrule
\end{tabular}
	\caption{Running times for different index implementations.}	
\label{tab:benchmark:indices}
\end{table} 	
