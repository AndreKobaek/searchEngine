\section{Software Design}

\subsection{Coupling}
When writing a program one should aim for low (or loose) coupling between classes. Low coupling means that classes are largely independent and communicate via a small well-defined interface \cite[p.259]{BK}. Hence a class should never depend on parts of another class that are not exposed via its interface. The interface of a class corresponds to its public methods and fields.
\\
In order to achieve low coupling in our project we have largely followed the structure of the program as it was handed out, since we agreed that it was designed in a reasonable way that enabled low coupling between the classes. Additionally, we followed the teachings of the course, keeping fields private.

 \begin{figure}[t]
	\centering
	\caption{UML Diagram for the class Website.}
	\label{fig:uml:single-class-website}
\end{figure}

 
 \subsubsection{A note about the package private modifier}
When working within a package, one could argue that all non-private fields (and methods) are part of the interface, since they are allowed to be accessed by other classes in the package. 
 
Package private access to a field in \code{Class} allows \code{OtherClass} to tinker with the state of \code{Class}, just as a setter method would.
A public getter method in \code{Class} on the other hand... would have exactly the same effect!
A getter method supplies the calling class \code{OtherClass} with a reference to the field, and thus changing the returned map will also change the state of \code{Class}.
 
For primitive types (and strings), getter methods doesn't expose state, but for fields of for example type \code{Map} \code{Set} or \code{List}, care should be taken if ones do not intend to expose the state.
In the cases above a package private modifier would not be a worse choice than the combination of private field and a getter method that returns the object \emph{reference}. To avoid exposing state one could choose to clone/copy the object and then return the clone/copy in the getter method.

A benefit of choosing the combination of a private field and a getter method is that we are able to change the behaviour of the getter method if we want to, for instance change it to return a (reference to a) clone/copy instead of returning (a reference to) the actual object. 


\subsection{Cohesion}
A program should aim for high cohesion. %input a reference here?
High cohesion means that a single method is responsible for a single task, and that a class has well defined area of responsibility. 

The class \code{QueryHandler} is responsible for handling queries, when it's method \code{getMatchingWebsites} is supplied with a query it fetches the matching websites. But should the \code{QueryHandler} also rank the sites it fetches via \code{getMatchingWebsites}? 

One argument for placing the ranking inside the \code{QueryHandler} is, that would be possible to avoid creating duplicate code for refining a query. But we concluded that the cohesion outweighed the implications of introducing duplicate code.

Therefore the task of ranking the websites found by the \code{QueryHandler} is assigned to the  \code{SearchEngine} which is already responsible for organizing the process of finding search results.

Cohesion is closely related to responsibility-driven design, which philosophy's main point is that "\textit{each class should be responsible for handling its own data}" %\citep{270}. This means that whatever information stored in a class is the class's own responsibility to provide or change. This ties into encapsulation as well. %PLEASE ADD TO THIS SECTION.

\subsection{Encapsulation}
Encapsulation is an approach to class interaction in object-oriented programming as well as a tool from the defensive programming toolbox. Encapsulation is an ideal way to fulfil the intention of low coupling. When an object is encapsulated the interface towards other packages is controlled by keeping the state of the object unexposed to other classes. This is not always possible, since reference type objects are mutable and can therefore be manipulated when access to the object is provided. The theoretical workaround is to only exposing clones of the reference type objects, which has the drawback of using additional memory. We decided that it would an unnecessary strong commitment to encapsulation that did not serve a purpose for all the internal interfaces within the \code{searchEngine} package.

\subsection{Inheritance} 
Inheritance is an architectural tool that helps to avoid duplicate code. By conceptually abstracting similar objects into a common object similarities can be abstracted into the common object, thus avoiding duplicate code. This property is used in the case of the \code{InvertedIndex} and the two different instances of it.

\subsection{Polymorphism} 

\subsection{Another design related concept}
something...
 
\subsection{Streams and Lambdas}
Maybe a section about streams and lambdas? When are they useful, an when are they not?


 
