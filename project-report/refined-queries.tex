\section{Refined Queries}

During the implementation of the inverted index, we add simple input validation to \code{QueryHandler} to only accept queries comprised of characters in the English alphabet, digits, dashes and underscores. This is done using the following regular expression: \code{\\\\b([-\\\\w]+)\\\\b}. Additionally, query strings are transformed to lowercase before doing a lookup.

For the refined queries, we build upon this functionality.

\subsection{Multiple Words}
First off, we add logic to handle multiple words in a search query. We need to return the websites containing all the words in the query string. Since we utilise \code{Pattern} and \code{Matcher}, the procedure is quite simple: If there is more than one match, the results from the first lookup is stored in a set and the results from the following lookups are retained from the same set. After the final lookup, a list is returned (since this is what \code{SearchEngine} and \code{WebApplication} expects).

Tests for the implementation are located in \code{QueryHandlerTest.java}.

\subsection{Merging via "OR"}
Second, we implement functionality to allow merging word sequences via "OR". The search query is split around the matches of the regular expression \code{\\\\bOR\\\\b}. The resulting sequences are stored in an array of strings, and for each of these strings a set of matching websites is found. The union of these sets form the response to the query. The nature of sets prevent duplicate elements. Finally, the results are added to a list which is returned.

Tests for the implementation are located in \code{QueryHandlerTest.java}.

\subsection{URL Filter}
Additionally, we added an URL filter to the \code{QueryHandler}. The intention of the URL filter is to let the user specify which websites he/she wants to receive results from. This is done by using the syntax \code{site:https://website.url/ } when writing a query. This functionality is implemented in the \code{getMathcingWebsites}, using, primarily, a regular expression: \code{^\\\\s*site:(\\\\S+)}. The captured URL string is then applied to verify the matching websites found by the method. Only the verified methods are returned.

The functionality is effectively useless on the available database files, since they all share the same domain which has a flat hierarchy. But the functionality is proved in the tests which are located in \code{QueryHandlerTest.java}.