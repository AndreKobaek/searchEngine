\documentclass[a4paper, oneside]{memoir}

% input encoding and language
\usepackage[utf8]{inputenc}
\usepackage[T1]{fontenc}
\usepackage[english]{babel}

% tables
\usepackage{booktabs}

% Colours and graphics
\usepackage{graphicx}
\graphicspath{ {.} }


\usepackage{color}  % perhaps \xcolor? what is the difference?
\definecolor{codegreen}{rgb}{0,0.6,0}
\definecolor{codegray}{rgb}{0.5,0.5,0.5}
\definecolor{codepurple}{rgb}{0.58,0,0.82}
\definecolor{backcolour}{rgb}{0.95,0.95,0.92}

% tikz
\usepackage{tikz}
\usepackage{aeguill}  % used by plantuml
\usetikzlibrary{calc} % used by plantuml

%% settings created and used by plantuml
% generated by Plantuml 1.2018.13      
\tikzset{
	href node/.style={
		alias=sourcenode,
		append after command={
			let \p1 = (sourcenode.north west),
			\p2=(sourcenode.south east),
			\n1={\x2-\x1},
			\n2={\y2-\y1} in
			node [inner sep=0pt, outer sep=0pt,anchor=north west,at=(\p1)] {\href{#1}{\XeTeXLinkBox{\phantom{\rule{\n1}{\n2}}}}}
			%xelatex needs \XeTeXLinkBox, won't create a link unless it
			%finds text --- rules don't work without \XeTeXLinkBox.
			%Still builds correctly with pdflatex and lualatex
		}
	}
}
\tikzset{
	hyperref node/.style={
		alias=sourcenode,
		append after command={
			let \p1 = (sourcenode.north west),
			\p2=(sourcenode.south east),
			\n1={\x2-\x1},
			\n2={\y2-\y1} in
			node [inner sep=0pt, outer sep=0pt,anchor=north west,at=(\p1)] {\hyperref [#1]{\XeTeXLinkBox{\phantom{\rule{\n1}{\n2}}}}}
			%xelatex needs \XeTeXLinkBox, won't create a link unless it
			%finds text --- rules don't work without \XeTeXLinkBox.
			%Still builds correctly with pdflatex and lualatex
		}
	}
}
\definecolor{plantucolor0000}{RGB}{254,254,206}
\definecolor{plantucolor0001}{RGB}{168,0,54}
\definecolor{plantucolor0002}{RGB}{173,209,178}
\definecolor{plantucolor0003}{RGB}{0,0,0}
\definecolor{plantucolor0004}{RGB}{200,41,48}
\definecolor{plantucolor0005}{RGB}{132,190,132}
\definecolor{plantucolor0006}{RGB}{3,128,72}




% Hyperlinked version
\usepackage[colorlinks]{hyperref}


% TODO in text
\usepackage{todonotes}


% Code formatting
\usepackage{listings}
\lstset{
	language=java,
	extendedchars=true,
	basicstyle=\footnotesize\ttfamily,
	showstringspaces=false,
	showspaces=false,
	numbers=left,
	numberstyle=\footnotesize,
	numbersep=9pt,
	tabsize=2,
	breaklines=true,
	showtabs=false,
	frame=single,
	extendedchars=false,
	inputencoding=utf8,
	captionpos=b
}

\lstdefinestyle{ipm-style}{
	backgroundcolor=\color{backcolour},   
	commentstyle=\color{codegreen},
	keywordstyle=\color{magenta},
	numberstyle=\tiny\color{codegray},
	stringstyle=\color{codepurple},
	basicstyle=\footnotesize,
	breakatwhitespace=false,         
	breaklines=true,                 
	captionpos=b,                    
	keepspaces=true,                 
	numbers=left,                    
	numbersep=5pt,                  
	showspaces=false,                
	showstringspaces=false,
	showtabs=false,                  
	tabsize=2
}
\lstset{style=ipm-style}

\newcommand{\code}[1]{\colorbox{backcolour}{\lstinline|#1|}}

% Bibliography, Index and Appendices 

\bibliographystyle{plainnat} % Use the plainnat style of referencing
%\usepackage{makeidx} % Used to generate the index
%\makeindex % Generate the index which is printed at the end of the document

